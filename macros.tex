%this file contains all the macros I need/want for my thesis.

%title info, replace with your version
\newcommand{\mytitle}{Standard Caps Title }
\newcommand{\mytitlecaps}{TITLE WITH ALL CAPS }

\newcommand{\mysubtitle}{Standard Caps Sub Title}     %you may or may not even want a subtitle
\newcommand{\mysubtitlecaps}{SUBTITLE WITH ALL CAPS}

%my name, normal and last, first
\newcommand{\me}{Your Full Name}
\newcommand{\melastfirst}{Last, First M.}

%current college and grad date information-------------------
\newcommand{\college}{The University of Montana}
\newcommand{\location}{Missoula, MT}
\newcommand{\mycurrentdegree}{Master of Science}
\newcommand{\mycurrentdegreeabbrev}{M.S.}
\newcommand{\mygradsemester}{Summer}
\newcommand{\mygradmonth}{August}
\newcommand{\mygradyear}{2015}
\newcommand{\mymajordept}{Computer Science}
%------------------------------------------------------------

%previous college and grad date information-----------------
\newcommand{\myprevdegreetwo}{Bachelor of Science}
\newcommand{\myprevcollegetwo}{The University of Montana}
\newcommand{\myprevlocationtwo}{Missoula, MT}
\newcommand{\myprevdegdatetwo}{2012}

\newcommand{\myprevdegree}{Associate of Science}
\newcommand{\myprevcollege}{The University of Montana}
\newcommand{\myprevlocation}{Helena, MT}
\newcommand{\myprevdegdate}{2007}
%-----------------------------------------------------------

%approval and committee members----------------------------
	\newcommand{\gradschooldean}{J.B. Alexander S. Ross}

	%committee chairman------------
	\newcommand{\chairname}{John Doe}
	\newcommand{\chairdept}{\mymajordept}
	%------------------------------
	
	%committee member #1-----------
	\newcommand{\committeememberonename}{Another Person}
	\newcommand{\committeememberonedept}{\mymajordept}
	%------------------------------
	
	%committee member #2 (outside dept)
	\newcommand{\committeemembertwoname}{Amazing Scientist}
	\newcommand{\committeemembertwodept}{Biological Sciences}

	% http://holben-lab.dbs.umt.edu/cv.php
	% http://etd.lib.umt.edu/theses/available/etd-05272008-134138/unrestricted/umi-umt-1066.pdf

	%------------------------------
%----------------------------------------------------------


%figure, table, etc... shortcuts---------------------------
%the tilde character prevents the wrapping of the command across
%two lines.  Any changes to UM's formatting can be accomplished
%by changing the text inside the {} here.
\newcommand{\sect}[1]{Section~\ref{#1}}
\newcommand{\eq}[1]{Equation~\eqref{#1}}
\newcommand{\fig}[1]{Figure~\ref{#1}}
\newcommand{\clis}[1]{Code Listing~\ref{#1}}
\newcommand{\tbl}[1]{Table~\ref{#1}}
\newcommand{\apdx}[1]{Appendix~\ref{#1}}
%---------------------------------------------------------

%---------------------------------------------------------
\usepackage{courier}
% Setup special formatting commands for things like 
% variables or system commands
% maybe italics, bold, or even change font to like courier
% \textit -> italics
% \textbf -> bold
% \underline -> underline
\newcommand{\var}[1]{\textit{#1}}

\newcommand{\sys}[1]{\textit{\underline{#1}}}

\newcommand{\func}[1]{\textbf{#1}}

\newcommand{\script}[1]{\texttt{\textbf{#1}}}

%macros to make addfigure (see below) work ---------------
\newcommand{\figformat}{png}	  %extension of graphics
\newcommand{\figdir}{figures/\figformat/}  %relative to thesis.tex
%my setup was ./figures/eps/*.eps (all fig files were eps)
%change yours accordingly, I strongly suggest keeping all
%figure files the same format
%---------------------------------------------------------

%single figure macro, adds to table of contents-----------
%and can have a separate name for caption and toc entry
%if no 3rd value, toc entry is same as caption
%otherwise, 3rd value is toc entry
% parameter 1 == file name (no extension
% parameter 2 == percent of text width (up to 1.0, 0.7 is good)
% parameter 3 == toc entry text (if blank, will use 
%                value for parameter 4
% parameter 4 == caption text
% parameter 5 == label for referencing 
\newcommand{\addfigure}[5]{%
\begin{figure}[htbp] %h!
\centering
\singlespacing
\footnotesize
\includegraphics*[width=#2\textwidth]{\figdir#1.\figformat}
\ifthenelse{\equal{#3}{}} %if 3 is blank
{% then
\isucaption[#4]{#4}
}%
{% else
\isucaption[#3]{#4}
}%
\label{#5}
\end{figure}
}%
%-------------------------------------------------------
\newcommand{\tblformat}{csv}
\newcommand{\tbldir}{tables/}

%This will blow up if your table is too wide for the document
%Tables are kind of a pain in LaTeX. My opinion is that making a nice
%table in excel or open office which can be turned into a jpg or png
%is often a better route.
\newcommand{\addtable}[2]{
  \csvautotabular{\tbldir#1/#2.\tblformat}
}
%-------------------------------------------------------

%code framing environment-------------------------------

%1 - language for syntax highlighting
%2 - source folder
%3 - filename
\newenvironment{framecode}[5]%
{
 %setup language specific syntax highlightning
 % see https://en.wikibooks.org/wiki/LaTeX/Source_Code_Listings#Supported_languages
 % for a list of language options
 \lstset{language=#1} 
 %\vspace{.5cm} 
 \begin{singlespace}
 \lstinputlisting[caption={#5}, label=#4]{\appdir#2/#3}
 \end{singlespace}
}{}


%------------------------------------------------------

\newcommand{\appdir}{source/}

%1 - caption as new section
%2 - filename
%3 - label
\newenvironment{dispcode}[3]%
{
\section{#1}\label{#3}
\begin{singlespace}
%\begin{footnotesize}
%\lstinputlisting[caption={\lstname}]{\appdir/#2} %use this line if you want list of listings to show the appendix code
\lstinputlisting{\appdir/#2}
%\verbatiminput{\appdir/#2}
%\end{footnotesize}
\end{singlespace}
}{}