%\twocolumn

%========================================================================
%Beginning of Chapter 2
%========================================================================

\chapter{LITERATURE REVIEW}\label{ch:overview}

This can be a hard section to write because it will force you to learn about the field that your research is a part of.
It's also a really good place to start on your thesis because the stuff you learn here will likely drive the rest of your research.

\section{Citations} \label{sec:citations}

This chapter should have lots of citations and possibly several sections \cite{biom} \cite{LDA} \cite{fasta}.
I would recommend using a tool like \sys{JabRef} to manage your citation database in \script{references.bib}.
JabRef will allow you to lookup and import citations from Search - Web Search (F5), which will open a panel on the left where you can pick a citation database like google scholar to search for journals by title.

You can check out the \emph{natbib} package for more options to the above macros, such as including text in the reference itself.

Check out the macro.tex file that has all the macros I hacked up for this thesis.  I created macros for inserting figures as well as referencing figures, tables, and equations.  Use them if you wish or strike out on your own.

\newpage
\section{Equations} \label{sec:equations}

An unnumbered equation array:

\begin{eqnarray*}
O_{1,max} &= &  8.0752 \quad \text{(Output 1's global maximum)} \\
O_{1,min} &= &-6.5466  \quad \text{(Output 1's global minimum)} 
\end{eqnarray*}

An in-line math environment:
\[
-6.5466 \leq O_1 \leq 8.0752
\]

An equation.  Note that it is split across several lines of text.
\begin{equation}
\label{eq:peaks}
\begin{split}
z = &3(1-x)^2e^{-(x^2) - (y+1)^2} - \ldots \\
&10(\frac{x}{5}-x^3 - y^5)e^{-x^2-y^2}- \ldots \\
&\frac{1}{3}e^{-(x+1)^2 - y^2}
\end{split}
\end{equation}

%========================================================================
%End of Chapter 2
%========================================================================

