%\twocolumn

\chapter{METHODS}\label{ch:methods}

The methods described below are roughly organized in the order they were performed during the case studies. 
Some methods are independent of the rest, so these methods are arranged at the end.

\section{Methods Syntax And Conventions}

Some of the special syntax conventions used in this document are as follows.

\begin{itemize}
\item \textit{Scientific Names} - capitalized and italicized.
\item \script{Scripts} - bold and Courier font.
\item \var{Variables} - italicized
\item \func{Functions} - bold
\item \sys{System Commands} - italicized and underlined
\end{itemize}

\clearpage %forces the latex engine to make a newpage and place any floating figures or tables BEFORE going past this point.

\section{Code With Syntax Highlighting} \label{sec:code}
  
The tsv file must also have lineage strings available, which can be created with the bash command in \clis{code:biom}.

% #1 the language to use for syntax highlighting
% #2 the name of the folder in source
% #3 the filename of the script
% #4 the label used to refer to this code elsewhere in the text
% #5 the caption used in the text and in the code listing near the table of contents
\framecode{bash}{snippets}{convertBiom.bash}
          {code:biom}{Convert Biom Matrix To TSV}

This process is necessary because QIIME's OTU matrix output is in the biom format \cite{biom}.
In order to convert a biom format matrix into a tsv matrix, a UNIX or Linux style operating system (OS) is recommended. 
Ubuntu 15.2 run from VirtualBox \cite{virtualbox} was used to test the \sys{biom} command successfully at the time of this writing. 
From a Linux based OS the \sys{biom} command can be installed in a terminal window as seen in \clis{code:pip}. 
In order for \clis{code:pip} to work, the \sys{pip} command needs to be available. 
Pip is the PyPA recommended tool for installing python packages, which makes it a good way to install many useful and important bioinformatics tools \cite{pip}.

\framecode{bash}{snippets}{install_pip.bash}
          {code:pip}{Install PIP}

\clearpage
\framecode{R}{snippets}{findParetoBoundary.R}
          {code:paretoboundary}{Find N-Dimensional Pareto Frontier}

%\clearpage
\framecode{R}{snippets}{paretoDomination.R}
          {code:paretodomination}{Check For Pareto Domination}

%%%%%%%%%%
% End of Chapter 3
%%%%%%%%%%
